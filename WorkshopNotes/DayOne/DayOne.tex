\documentclass{ximera}

\title{Day One 1275EN Workshop}
\author{K. Andrew Parker}
\begin{abstract}
Introduction to 1275EN Pedgagogical Strategies
\end{abstract}

\maketitle

Think about our audience: What do we know about them?
\begin{itemize}
\item incoming students - first-years
\item continuing from developmental math
\item possibly repeaters
\end{itemize}

Who places into this class?
\begin{itemize}
\item exempt from remedial
\item poor scores on CUNY placement exam
\end{itemize}

Exempt from remedial => at some point they've learned this material already. \\

Poor placement scores => low retention and/or shallow comprehension \\

Conclusion: They've learned the content, but they haven't understood it
and/or integrated the algorithms they've been taught with the broader
mathematical concepts.\\

Alternatively, students possibly have decent comprehension and simply did not 
adequately review for the placement exam. \\

Regardless, students need to be pushed to see how the topics fit into a big-
picture view of mathematics. Push them away from a procedural knowledge of
math, and encourage deeper exploration.\\

\\

What can we expect from students who are continuing from developmental math? \\

Varied level of effort may have been required. They haven't seen any "new" 
mathematics for years. (They've been repeatedly covering the same basic algebra
topics.)\\

Again, more emphasis on deeper exploration of topics, and a focus on developing
"softer" competencies: study habits, approaching "scary" problems, persistence...\\

\\ 

What can we expect from repeaters?\\
\begin{itemize}
\item They're the most diverse group
\item Various reasons for failure
\item As with other groups, haven't seen "new" content in a long time.
\item Use a similar approach as the students continuing from developmental.
\item Extra attention as these students have taken a confidence-hit from failure.
\end{itemize}

\end{document}

