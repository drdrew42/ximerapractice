\documentclass{ximera}

\title{Day One 1275EN Workshop}
\author{K. Andrew Parker}
\begin{document}
\begin{abstract}
Getting the lay of the land in MAT1275EN
\end{abstract}

\maketitle

Successful strategies for instruction depend heavily on the characteristics of the students to whom we are teaching.\\

Before we can even begin considering teaching strategies, it is crucial that we understand our audience! \\


\section{Categorizing students}

\begin{itemize}
\item incoming students placing into MAT1175 - first-years
\item continuing from developmental math
\item possibly repeaters
\end{itemize}

\section{Who places into this class?}

\begin{itemize}
\item exempt from remedial
\item they placed into MAT1175
\item poor scores on CUNY placement exam
\item "not ready" for MAT1275, but WHY?
\end{itemize}

\begin{lemma}
Students exempt from remedial *must have,* at some point, learned this material already.
\end{lemma}

\begin{lemma}
Exempt students with poor placement scores have low retention and/or shallow comprehension.
\end{lemma}

\begin{theorem}
Exempt students with poor placement scores have learned the content, 
but they haven't understood it beyond a baseline, procedural level 
and/or integrated the algorithms they were taught into a broader perspective 
of mathematical concepts in general.
\end{theorem}

\begin{corollary}
These students possibly have decent comprehension and simply did not 
adequately review for the placement exam. \\
\end{corollary}

Regardless, students need to be pushed to see how the topics fit into a big-
picture view of mathematics. Push them away from a procedural knowledge of
math, and encourage deeper exploration.\\

\section{Students Continuing from Developmental Math}

\begin{itemize}
\item Varied level of effort may have been required. 
\item They haven't seen any "new" mathematics for years. (They've been repeatedly covering the same basic algebra
topics.)
\end{itemize}

Again, more emphasis on deeper exploration of topics, and a focus on developing
"softer" competencies: study habits, approaching "scary" problems, persistence...\\


\section{What can we expect from repeaters?}

\begin{itemize}
\item They're the most diverse group
\item Various reasons for failure
\item As with other groups, haven't seen "new" content in a long time.
\item Use a similar approach as the students continuing from developmental.
\item Extra attention as these students have taken a confidence-hit from failure.
\end{itemize}

\end{document}

