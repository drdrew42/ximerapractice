\documentclass{ximera}

\title{MAT1375: Absolute Value Inequalities}
\author{K. Andrew Parker}
\begin{document}
\begin{abstract}
Interval notation and the solution of absolute value inequalities
\end{abstract}

\maketitle

\section{Solving Absolute Value Inequalities}

\subsection{Solving Absolute Value EQUALITIES}

The absolute value of $x$ can be equal to any \emph{positive} real number.

In fact, there are two possible values of $x$ that solve the equation $|x|=a$ (for any \emph{positive} value, $a$).

See the following illustration:

Investigate how the absolute value of $x$ (the "V" shape) compares to the value of $a$ (the horizontal line).

\begin{center}
\geogebra{vGups2D3}{650}{450}
\end{center}

\subsection{Switching over to INEQUALITIES}

\begin{center}
\geogebra{qCcVzAKB}{650}{500}
\end{center}

Use this illustration to answer the following questions:

\begin{itemize}
 \item What does the solution to $|x| = a$ look like?
 \item What does the solution to $|x| > a$ look like?
 \item What changes when we include equality? $|x| \geq a$?
 \item What does the solution to $|x| < a$ look like?
 \item What changes when we include equality? $|x| \leq a$?
\end{itemize}

\begin{center}
\geogebra{ZSNtTT5D}{700}{400}
\end{center}

\end{document}

