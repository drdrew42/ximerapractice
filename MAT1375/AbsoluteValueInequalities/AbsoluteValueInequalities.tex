\documentclass{ximera}

\title{MAT1375: Absolute Value Inequalities}
\author{K. Andrew Parker}
\begin{document}
\begin{abstract}
Interval notation and the solution of absolute value inequalities
\end{abstract}

\maketitle

\graph{y=abs(x)}

\geogebra{ZSNtTT5D}{600}{400}

\section{Classifying students}

\begin{enumerate}
\item incoming students placing into MAT1175 - first-years
\item students continuing from developmental math
\item possibly repeaters - failed 1175 or 1275EN
\end{enumerate}

\subsection{Who is placing into this class?}

\begin{itemize}
\item they are exempt from remedial
\item they placed into MAT1175
\item poor scores on CUNY placement exam
\item "not ready" for MAT1275, but WHY?
\end{itemize}

What can we determine from these facts?

\begin{lemma}
These students are exempt from remedial. Therefore, they \emph{must have,} at some point, learned this material already.
\end{lemma}

\begin{lemma}
Exempt students with poor placement scores have low retention and/or shallow comprehension.
\end{lemma}

\begin{theorem}
Exempt students with poor placement scores have learned the content, 
but they haven't understood it beyond a baseline, procedural level 
and/or integrated the algorithms they were taught into a broader perspective 
of mathematical concepts in general.
\end{theorem}

\begin{corollary}
These students potentially have decent comprehension and simply did not 
adequately review for the placement exam. \\
\end{corollary}

Regardless, students need to be pushed to see how the topics fit into a big-
picture view of mathematics. Push them away from a procedural knowledge of
math, and encourage deeper exploration.\\

\subsection{Students Continuing from Developmental Math}

\begin{itemize}
\item Varied level of effort may have been required. 
\item They haven't seen any "new" mathematics for years. (They've been repeatedly covering the same basic algebra
topics.)
\end{itemize}

Again, more emphasis on deeper exploration of topics, and a focus on developing
"softer" competencies: study habits, approaching "scary" problems, persistence...\\


\subsection{What can we expect from repeaters?}

\begin{itemize}
\item They're the most diverse group
\item Various reasons for failure
\item As with other groups, haven't seen "new" content in a long time.
\item Use a similar approach as the students continuing from developmental.
\item Extra attention as these students have taken a confidence-hit from failure.
\end{itemize}

\section{Conclusions}

How should we adapt our instructional strategies to accommodate these findings?

\subsection{Instructor Strategies}

\begin{itemize}
 \item Students have "seen this before". 
 \item As a result, we must be conscientious of we how cover topics.
 \item If students suspect that you're teaching something they "already know", they may tune out - missing relevant content that could help them fit this piece into the larger puzzle.
 \item Use "alternate" methods to solve problems.
 \item Avoid teaching "shortcuts". 
 \item Be diligent about writing out \emph{every} step. 
 \item Focus on the basics, showing how the "new" content is an abstraction of prior arithmetic knowledge.
 \item Let students wrestle with the tough problems in the WeBWorK.
\end{itemize}

\subsection{Active Learning}

\begin{itemize}
 \item Begin using group work right away.
 \item Have students work together in groups of 3-4. 
 \item Assign their groups the first several times. 
 \item Change up their groups each time. 
 \item Require that they present their findings in the \emph{next} class. 
 \item Force them to communicate outside of class. (Share contact info)
\end{itemize}

\subsection{Soft Competencies}

\begin{itemize}
 \item Early intervention for insufficient progress.
 \item Consistent and frequent assessment.
 \item Emphasize the availability of WeBWorK features:
 \begin{itemize}
  \item Linked videos
  \item Hints and Solutions
  \item OpenLab Q\&A Site
  \item Show Me Another
 \end{itemize}
 \item "Entering answers in WeBWorK" for Computer Science majors
 \item Avoiding compunded error (aka accuracy) for Architecture/Engineering majors
\end{itemize}

\end{document}

